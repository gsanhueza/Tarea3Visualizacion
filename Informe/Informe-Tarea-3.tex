\documentclass[letterpaper,10pt]{article}

\usepackage[utf8]{inputenc}
\usepackage[spanish]{babel}
\usepackage{fontenc}
\usepackage[dvipdfmx]{graphicx}
\usepackage{bmpsize,wrapfig,xcolor}
\usepackage{fullpage}
\usepackage[hidelinks]{hyperref}

% Para evitar que se indente solo a cada rato
\setlength\parindent{0pt}

\begin{document}
	\begin{titlepage}

		\begin{wrapfigure}{R}{0.3\textwidth}
			\includegraphics[width=0.3\textwidth]{logoFCFM.png}
		\end{wrapfigure}

		\noindent \phantom - % "Hax" para que quede alineada la imagen con el texto

		Universidad de Chile

		Facultad de Ciencias Físicas y Matemáticas

		Depto. de Ciencias de la Computación

		CC5208 - Visualización de Información

		\vfill

		\begin{center}
			\begin{Huge}
				{\textbf{Tarea 3}}
			\end{Huge}

			\begin{large}
				Layout del proyecto
			\end{large}

		\end{center}

		\vfill

		\begin{flushright}
			\begin{tabular}{lll}
				Integrantes	&:	& Américo Ferrada\\
						&:	& Gabriel Sanhueza\\
				Profesor	&:	& Javier Bustos\\
				Ayudante	&:	& Diego Madariaga\\
			\end{tabular}
		\end{flushright}

	\end{titlepage}

	% % % % % % % % % % % % % % % % % % % % % % % % % % % % % % % % % % % % % % % % % % % % % % % % % % % % % % % % % % % % % % % % % % % % % % % % % % % % % % % % % % % % % % % % % %
	\newpage
	% % % % % % % % % % % % % % % % % % % % % % % % % % % % % % % % % % % % % % % % % % % % % % % % % % % % % % % % % % % % % % % % % % % % % % % % % % % % % % % % % % % % % % % % % %

	\tableofcontents

	% % % % % % % % % % % % % % % % % % % % % % % % % % % % % % % % % % % % % % % % % % % % % % % % % % % % % % % % % % % % % % % % % % % % % % % % % % % % % % % % % % % % % % % % % %
	\newpage
	% % % % % % % % % % % % % % % % % % % % % % % % % % % % % % % % % % % % % % % % % % % % % % % % % % % % % % % % % % % % % % % % % % % % % % % % % % % % % % % % % % % % % % % % % %

	\begin{large}
		{\textbf{BORRAR ESTO}}
	\end{large}

	En esta tarea debe enviar el diseño del layout para los datos de su proyecto. Es la última oportunidad que tiene para cambiar su dataset.

	Se espera que usted:

	\begin{enumerate}
		\item ** DONE: Haga un análisis de los tipos de datos que posee (ordinales, categóricos, cuantitativos, etc) y sus relaciones. **
		\item Plantee las preguntas que desea su visualización responda.
		\item Haga un mapeo de sus datos a los distintos gráficos que servirán para responder 2.
		\item Diseñe (usando lápiz/papel, paint, d3, tableau, lo que le sea más fácil) el layout de su visualización, textos de apoyo, y su paleta de colores,
		justificando sus decisiones de diseño desde el punto de vista cognitivo y con los fundamentos de información visual.
	\end{enumerate}

	Recuerde que hoy en clases se informó que sus datos y visualizaciones deben quedar en un repositorio de GitHub. Si usted no tiene uno propio debe avisarle al profesor para que le cree un repositorio.

	% % % % % % % % % % % % % % % % % % % % % % % % % % % % % % % % % % % % % % % % % % % % % % % % % % % % % % % % % % % % % % % % % % % % % % % % % % % % % % % % % % % % % % % % % %
	\newpage
	% % % % % % % % % % % % % % % % % % % % % % % % % % % % % % % % % % % % % % % % % % % % % % % % % % % % % % % % % % % % % % % % % % % % % % % % % % % % % % % % % % % % % % % % % %

	\section{Análisis}

	Existen 10 columnas de datos en total:

	\begin{itemize}
		\item Nombre: Aleatorio
		\item Id: Ordinal
		\item Tipo: Categórico
		\item Clase: Categórico
		\item Masa: Ordinal

		\item Caído/Encontrado: Categórico
		\item Año: Ordinal
		\item Latitud: Ordinal
		\item Longitud: Ordinal
		\item Geolocalización: Ordinal (Tupla)
	\end{itemize}

	Nuestros datos se relacionan de la siguiente manera:

	\begin{itemize}
		\item Tamaño y tipo del meteorito: Cada tipo de meteorito depende de su tamaño.
		\item Cantidad de meteoritos y año encontrado: Entre más reciente, mayor es la cantidad de meteoritos encontrados.
		\item Tamaño de los meteoritos y geolocalización: Se encuentra mayor cantidad de meteoritos pequeños en los polos.
		\item Nombre propio y tamaño del meteorito: Los meteoritos de tamaño no-despreciable tienen nombre propio (Ej.: Alessandria) v/s los más pequeños (Ej.: Yamato 983824)
	\end{itemize}

	% % % % % % % % % % % % % % % % % % % % % % % % % % % % % % % % % % % % % % % % % % % % % % % % % % % % % % % % % % % % % % % % % % % % % % % % % % % % % % % % % % % % % % % % % %
	\newpage
	% % % % % % % % % % % % % % % % % % % % % % % % % % % % % % % % % % % % % % % % % % % % % % % % % % % % % % % % % % % % % % % % % % % % % % % % % % % % % % % % % % % % % % % % % %

	\section{Planteamiento de preguntas}

	Preguntas que esperamos responder con nuestra visualización:


	\begin{enumerate}
		\item ¿Dónde se encuentran más meteoritos en función del peso?
		\item ¿Cuál es el tipo de meteoritos más frecuente? % Contador vs Tipos
		\item Respecto al peso, ¿cuáles son los tipos de meteoritos encontrados? % Peso vs Tipo

		\item ¿Cuál es el peso de meteorito más frecuente?
		\item ¿Cuáles son los 10 meteoritos más pesados encontrados?
		\item ¿Cuáles son los 10 meteoritos más livianos encontrados?

		\item ¿En qué año se encontraron más meteoritos?
		\item ¿En qué pais se han encontrado más meteoritos?
		\item ¿Cuál es el tipo de meteorito más frecuente en función de su geolocalización?
	\end{enumerate}

	\section{Mapeo datos a gráficos}

	Como relacionar los datos para responder las preguntas anteriores:

	\begin{enumerate}
		\item Basta mapear los datos (geolocalización) en un mapa mundial, agregando un filtro en función del peso y usando distintos colores para cada segmento.
		\item En un gráfico de barra, por cada tipo contar sus apariciones.
		\item En un gráfico de barra, tomando solo un segmento de los datos en función del peso.

		\item En un grafico de barra, ordenar los meteoritos en función del peso y contar sus ocurrencias.
		\item Se usa gráfico anterior.
		\item Se usa gráfico anterior.

		\item En un grafico de barra, contar apariciones de meteoritos en función del año.
		\item Dado un mapa mundial, utilizar un gradiente de color en función del número de meteoritos caídos por país.
		\item Dado un mapa mundial, utilizar un gradiente de color en función del número de meteoritos caídos por tipo.
	\end{enumerate}

	% % % % % % % % % % % % % % % % % % % % % % % % % % % % % % % % % % % % % % % % % % % % % % % % % % % % % % % % % % % % % % % % % % % % % % % % % % % % % % % % % % % % % % % % % %
	\newpage
	% % % % % % % % % % % % % % % % % % % % % % % % % % % % % % % % % % % % % % % % % % % % % % % % % % % % % % % % % % % % % % % % % % % % % % % % % % % % % % % % % % % % % % % % % %

	\section{Diseño}
	% TODO Diseño

	TODO

	% 	Paleta de colores: http://paletton.com/#uid=50K0D0kS-C8tqT2FFTwQPquQols

\end{document}
