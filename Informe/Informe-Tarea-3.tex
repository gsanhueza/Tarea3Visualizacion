\documentclass[letterpaper,10pt]{article}

\usepackage[utf8]{inputenc}
\usepackage[spanish]{babel}
\usepackage{fontenc}
\usepackage[dvipdfmx]{graphicx}
\usepackage{bmpsize,wrapfig,xcolor}
\usepackage{fullpage}
\usepackage[hidelinks]{hyperref}

% Para evitar que se indente solo a cada rato
\setlength\parindent{0pt}

\begin{document}
	\begin{titlepage}

		\begin{wrapfigure}{R}{0.3\textwidth}
			\includegraphics[width=0.3\textwidth]{logoFCFM.png}
		\end{wrapfigure}

		\noindent \phantom - % "Hax" para que quede alineada la imagen con el texto

		Universidad de Chile

		Facultad de Ciencias Físicas y Matemáticas

		Depto. de Ciencias de la Computación

		CC5208 - Visualización de Información

		\vfill

		\begin{center}
			\begin{Huge}
				{\textbf{Tarea 3}}
			\end{Huge}

			\begin{large}
				Layout del proyecto
			\end{large}

		\end{center}

		\vfill

		\begin{flushright}
			\begin{tabular}{lll}
				Integrantes	&:	& Américo Ferrada\\
						&:	& Gabriel Sanhueza\\
				Profesor	&:	& Javier Bustos\\
				Ayudante	&:	& Diego Madariaga\\
			\end{tabular}
		\end{flushright}

	\end{titlepage}

	% % % % % % % % % % % % % % % % % % % % % % % % % % % % % % % % % % % % % % % % % % % % % % % % % % % % % % % % % % % % % % % % % % % % % % % % % % % % % % % % % % % % % % % % % %
	\newpage
	% % % % % % % % % % % % % % % % % % % % % % % % % % % % % % % % % % % % % % % % % % % % % % % % % % % % % % % % % % % % % % % % % % % % % % % % % % % % % % % % % % % % % % % % % %

	\tableofcontents

	% % % % % % % % % % % % % % % % % % % % % % % % % % % % % % % % % % % % % % % % % % % % % % % % % % % % % % % % % % % % % % % % % % % % % % % % % % % % % % % % % % % % % % % % % %
	\newpage
	% % % % % % % % % % % % % % % % % % % % % % % % % % % % % % % % % % % % % % % % % % % % % % % % % % % % % % % % % % % % % % % % % % % % % % % % % % % % % % % % % % % % % % % % % %

	\begin{large}
		{\textbf{BORRAR ESTO}}
	\end{large}

	En esta tarea debe enviar el diseño del layout para los datos de su proyecto. Es la última oportunidad que tiene para cambiar su dataset.

	Se espera que usted:

	\begin{enumerate}
		\item Haga un análisis de los tipos de datos que posee (ordinales, categóricos, cuantitativos, etc) y sus relaciones.
		\item Plantee las preguntas que desea su visualización responda.
		\item Haga un mapeo de sus datos a los distintos gráficos que servirán para responder 2.
		\item Diseñe (usando lápiz/papel, paint, d3, tableau, lo que le sea más fácil) el layout de su visualización, textos de apoyo, y su paleta de colores,
		justificando sus decisiones de diseño desde el punto de vista cognitivo y con los fundamentos de información visual.
	\end{enumerate}

	Recuerde que hoy en clases se informó que sus datos y visualizaciones deben quedar en un repositorio de GitHub. Si usted no tiene uno propio debe avisarle al profesor para que le cree un repositorio.

	% % % % % % % % % % % % % % % % % % % % % % % % % % % % % % % % % % % % % % % % % % % % % % % % % % % % % % % % % % % % % % % % % % % % % % % % % % % % % % % % % % % % % % % % % %
	\newpage
	% % % % % % % % % % % % % % % % % % % % % % % % % % % % % % % % % % % % % % % % % % % % % % % % % % % % % % % % % % % % % % % % % % % % % % % % % % % % % % % % % % % % % % % % % %

	\section{Análisis}
	% TODO Análisis

	TODO

	\section{Planteamiento de preguntas}
	% TODO Planteamiento de preguntas

	TODO

	\section{Mapeo datos a gráficos}
	% TODO Mapeo datos -> gráficos

	TODO

	\section{Diseño}
	% TODO Diseño

	TODO

\end{document}
